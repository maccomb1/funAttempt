\documentclass{ximera}

\title{Intermediate Value Theorem}
\author{Ryan Maccombs}
%\license{CC: 0}         % replace with an appropriate license, or set it in xmPreamble

\begin{document}
%\begin{abstract}
%    A simple Ximera activity.
%\end{abstract}
\maketitle
\label{xim:aFirstActivity}

Perhaps the most natural setting for Ximera content is that of a
\textit{worksheet}. This is some document that may contain discussion as well
as questions that check understanding.

Ximera comes pre-equipped with many environments.  If you are ever curious about
the source code, you can visit this source at

\section{A basic use case}
We use\verb|\begin{definition}| for definitions and \verb|\begin{question}| for
questions. Since Ximera provides immediate feedback, we suggest following
definitions like this one by a quick question. Here's an example:

\begin{definition}\label{def:absolute_value}
    The \textbf{absolute value} of a real number $a$, denoted by $|a|$, is
    \[
        |a| = \begin{cases}
            a  & \text{if $a \geq 0$} \\
            -a & \text{if $a<0$.}
        \end{cases}
    \]
\end{definition}
Now students can check their understanding:
\begin{question}
    Evaluate the following:
    \begin{enumerate}
        \item $|2-5| = \answer{3}$
        \item $|5-2| = \answer{3}$
        \item $|5-\sqrt{2}| = \answer{3.58578643763}$
        \item $|5-\sqrt{2}| = \answer{5-\sqrt{2}}$
    \end{enumerate}
\end{question}

To see why there are two versions of the $|5-\sqrt{2}|$ question, view the source code for this question by appending \verb|.tex| to the end of the URL.

\section{A paradox}


Here's something fun

\begin{paradox}[$0=1$] Let $x=y$ and write
\begin{align}
        x^2    & = xy       \\
    x^2 - y^2  & = xy - y^2 \\
    (x-y)(x+y) & = (x-y)y   \\
       (x+y)   & = y        \\
         2y    & = y        \\
          2    & =1.
\end{align}
Where is the mistake in the work above?
\begin{prompt} %% The content within prompt is normally not shown in a PDF or physical handout, as not relevant.
\[
\text{Between line }\answer{3} \text{ and line }\answer{4}.
\]
\end{prompt}
\end{paradox}


\section{Basic exercises}

After that, you might want to have some exercises.
You will not find any inspiration in the above \hyperref[def:absolute_value]{definition of the absolute value}.

\begin{exercise}
    Let $x$ be the number of people
    out of $100$ that LOVE Ximera.

    Find the value of $x$.
    \[
        x = \answer{100}
    \]
\end{exercise}

\end{document}
